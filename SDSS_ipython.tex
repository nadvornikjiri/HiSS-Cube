\section{SDSS Uncertainty Image+Spectral Cube
construction}\label{sdss-uncertainty-imagespectral-cube-construction}

In this notebook we show the methods in detail how do we construct the
uncertainty cube from SDSS Spectra and images. The data used are DR14
images and spectra from Stripe 82.

\begin{lstlisting}[language=Python]
import os
import sys
module_path = os.path.abspath(os.path.join('..'))
if module_path not in sys.path:
    sys.path.append(module_path)
    
from hisscube.utils.photometry import Photometry
from hisscube.utils.config import Config



from astropy.time import Time
from pathlib import Path
from matplotlib.colors import LogNorm
from importlib import reload
import h5py
import fitsio
from scipy import ndimage
import matplotlib.pyplot as plt
import numpy as np
import timeit
#%matplotlib notebookQQ


photometry = Photometry(Config())

test_image = "../data/raw/images/301/4797/1/frame-g-004797-1-0019.fits.bz2"    
test_spectrum = "../data/raw/spectra/blue_star.fits"

test_images_small = "frame-*-002820-3-0122.fits.bz2"
test_spectra_small = "*.fits"
test_images = "*.fits.bz2"
test_spectra = "*.fits"

img_header, img_data, file = photometry.read_fits_file(test_image)
spec_data, spec_header = photometry.read_spectrum(test_spectrum)
#print(img_header)
t = Time(img_header["DATE-OBS"], format='isot', scale='tai')

font = {'weight' : 'bold',
        'size'   : 24}

plt.rc('font', **font)
\end{lstlisting}

\section{Extracting uncertainty for image
measurements.}\label{extracting-uncertainty-for-image-measurements.}

In this step we extract the uncertainty for an SDSS image and plot all
the intermediate by-products below. Steps involved: 1. Reading sky
background from the FITS extension. 2. Interpolating the sky background
to the image resolution. 3. Reading calibration vector and extending it
to whole image (1D calibration vector works because of the way how SDSS
continuously reads out the image). 4. Subtracting sky background and
image calibration from the calibrated image, thus producing the Image
before calibration in Detector Data Numbers. 5. Showing how we can
calculate number of electrons from the data numbers - we don't need
these for our error calculation. 6. Calculating Errors in the Data
Numbers 7. Converting the Data Number errors to Nanomaggies (this is
what we store in the uncertainty cube) 8. Showing the error to signal
ratio - this plot demonstrates what we would expect - for brighter areas
of the image we are more certain about the measurements than for dark
areas.

\begin{lstlisting}[language=Python]
from numpy import inf
from matplotlib import colors

def draw_plots(img, img_err, allsky, simg, cimg, dn, nelec, dn_err, err_to_signal):
    plt.figure(figsize=(30,20))
    plt.imshow(allsky, cmap='afmhot', norm=LogNorm())
    plt.colorbar()
    plt.title("Sky background small")

    plt.figure(figsize=(30,20))
    plt.imshow(simg, cmap='afmhot', norm=LogNorm())
    plt.colorbar()
    plt.title("Sky background")

    plt.figure(figsize=(30,20))
    plt.imshow(img, cmap='afmhot', norm=LogNorm())
    plt.colorbar()
    plt.title("Calibrated image")

    plt.figure(figsize=(30,20))
    plt.imshow(dn, cmap='afmhot', norm=LogNorm())
    plt.colorbar()
    plt.title("Image before calibration in Data Numbers")

    plt.figure(figsize=(30,20))
    plt.imshow(nelec, cmap='afmhot', norm=LogNorm())
    plt.colorbar()
    plt.title("Number of photo-electrons")

    plt.figure(figsize=(30,20))
    plt.imshow(dn_err, cmap='afmhot', norm=LogNorm())
    plt.colorbar()
    plt.title("Errors in the Data Numbers")

    plt.figure(figsize=(30,20))
    plt.imshow(img_err, cmap='afmhot', norm=LogNorm())
    plt.colorbar()
    plt.title("Errors in nanomaggies")
    
    
    plt.figure(figsize=(30,20))
    plt.imshow(err_to_signal, cmap='afmhot', norm=LogNorm())
    plt.colorbar()
    plt.title("Uncertainty in signal - Uncertainty to signal ratio")
    

def get_image_with_errors(fitsPath):
    with fitsio.FITS(fitsPath) as f:
        start_time = timeit.default_timer()
        fits_header = f[0].read_header()
        img = f[0].read()
        x_size = fits_header["NAXIS1"]
        y_size = fits_header["NAXIS2"]
        camcol = fits_header['CAMCOL']
        run = fits_header['run']
        band = fits_header['filter']    
        allsky = f[2]['allsky'].read()[0]
        xinterp = f[2]['xinterp'].read()[0]
        yinterp = f[2]['yinterp'].read()[0]
        
        elapsed1 = timeit.default_timer() - start_time
        
        #print(elapsed1)


        gain = float(photometry.get_ccd_gain(camcol, run, band))
        darkVariance = float(photometry.get_dark_variance(camcol, run, band))
        elapsed2 = timeit.default_timer() - start_time
        
        #print(elapsed2)
        
        grid_x, grid_y = np.meshgrid(xinterp, yinterp, copy=False)
        elapsed3 = timeit.default_timer() - start_time
        
        #print(elapsed3)
        #interpolating sky background from 256x196 small image included in SDSS
        simg = ndimage.map_coordinates(allsky, (grid_y, grid_x), order = 1, mode="nearest")
        
        elapsed4 = timeit.default_timer() - start_time
        #print(elapsed4)

        calib = f[1].read()

        cimg = np.tile(calib, (y_size, 1)) #calibration image constructed from calibration vector

        dn = img / cimg + simg #data numbers detected originally by the detector 
        

        nelec = dn*gain #number of electrons

        dn_err = np.sqrt(dn/gain + darkVariance) #errors in data numbers (sigma)

        img_err= dn_err*cimg  #errors in nanomaggies (sigma)
        
        img_phys_flux = img

        err_to_signal = img_err/ img
        err_to_signal[err_to_signal == inf] = 0
        elapsed5 = timeit.default_timer() - start_time
        #print(elapsed5)
        
        

        
        return img, img_err, allsky, simg, cimg, dn, nelec, dn_err, err_to_signal
    
    
img, img_err, allsky, simg, cimg, dn, nelec, dn_err, err_to_signal = get_image_with_errors(test_image)

draw_plots(img, img_err, allsky, simg, cimg, dn, nelec, dn_err, err_to_signal)
\end{lstlisting}

\begin{lstlisting}
/tmp/ipykernel_33574/331323260.py:97: RuntimeWarning: divide by zero encountered in divide
  err_to_signal = img_err/ img
\end{lstlisting}

\begin{figure}
\centering
\includegraphics{SDSS_cube_files/SDSS_cube_3_1.png}
\caption{png}
\end{figure}

\begin{figure}
\centering
\includegraphics{SDSS_cube_files/SDSS_cube_3_2.png}
\caption{png}
\end{figure}

\begin{figure}
\centering
\includegraphics{SDSS_cube_files/SDSS_cube_3_3.png}
\caption{png}
\end{figure}

\begin{figure}
\centering
\includegraphics{SDSS_cube_files/SDSS_cube_3_4.png}
\caption{png}
\end{figure}

\begin{figure}
\centering
\includegraphics{SDSS_cube_files/SDSS_cube_3_5.png}
\caption{png}
\end{figure}

\begin{figure}
\centering
\includegraphics{SDSS_cube_files/SDSS_cube_3_6.png}
\caption{png}
\end{figure}

\begin{figure}
\centering
\includegraphics{SDSS_cube_files/SDSS_cube_3_7.png}
\caption{png}
\end{figure}

\begin{figure}
\centering
\includegraphics{SDSS_cube_files/SDSS_cube_3_8.png}
\caption{png}
\end{figure}

\section{Extracting spectrum errors}\label{extracting-spectrum-errors}

For the spectrum the SDSS has made life much more easy for us as the
inverse variance of the measurement is already stored in the fits files.
We just need to convert both to nano-maggies to be able to compare it
with image photometry, as it is in `1E-17 erg/cm\^{}2/s/Ang' units.

\begin{lstlisting}[language=Python]
def plot_spec(fits_path):
    with fitsio.FITS(fits_path) as hdul:
        data = hdul[1].read()
        flux = data["flux"] * 1e-17
        sigma =  np.sqrt(np.divide(1,data["ivar"])) * 1e-17
        wl = np.power(10, data["loglam"])
        print(data["loglam"])
        plt.figure(figsize=(30,20))
        ax = plt.axes(xlabel="Wavelength [Angstrem]", ylabel="Flux [ erg/cm^2/s/Ang]")
        ax.plot(wl, flux)
        ax.fill_between(wl, flux - sigma, flux + sigma, color="orange")
        return wl, flux, sigma

spec_wl, spec_flux, spec_sigma = plot_spec(test_spectrum)
\end{lstlisting}

\begin{lstlisting}
[3.5804 3.5805 3.5806 ... 3.9643 3.9644 3.9645]
\end{lstlisting}

\begin{figure}
\centering
\includegraphics{SDSS_cube_files/SDSS_cube_5_1.png}
\caption{png}
\end{figure}

\section{Getting Filter transmission
curves}\label{getting-filter-transmission-curves}

We are not able to compare directly spectroscopic measurements with
photometric ones. However, we can approximate this by applying the
filter transmission curves to the spectrum and thus simulate values that
would be measured to photon counts reduced by filter transmission
function.

Below we show how the transmission curve looks like throught the whole
spectrograph range. We take the transmission curves for individual
filters and take the maximum transmission ratio for each wavelength,
which essentially gets rid of the filter overlaps. We can do this
because the image coordinates are the mid\_points of those filters which
are no in the overlapping regions anyway, so there will be no ambiguity
to which filter should be applied in those regions.

\begin{lstlisting}[language=Python]
u = photometry.transmission_curves["u"]
g = photometry.transmission_curves["g"]
r = photometry.transmission_curves["r"]
i = photometry.transmission_curves["i"]
z = photometry.transmission_curves["z"]

merged_transmission_curve = photometry.merged_transmission_curve

wl, band_ratio = zip(*list(merged_transmission_curve.items()))
band, ratio = zip(*list(band_ratio))
u_wl, u_ratio = zip(*list(u.items()))
g_wl, g_ratio = zip(*list(g.items()))
r_wl, r_ratio = zip(*list(r.items()))
i_wl, i_ratio = zip(*list(i.items()))
z_wl, z_ratio = zip(*list(z.items()))

plt.rc('font', **font)

plt.figure(figsize=(30,20))
ax_u = plt.axes(xlabel="Wavelength [Angstrem]", ylabel="Transmission ratio")
ax_u = ax_u.plot(u_wl, u_ratio, color="blue")
ax_g = plt.plot(g_wl, g_ratio, color="green")
ax_r = plt.plot(r_wl, r_ratio, color="red")
ax_i = plt.plot(i_wl, i_ratio, color="#BF0000")
ax_z = plt.plot(z_wl, z_ratio, color="grey")
plt.title("Individual transmission curves for UGRIZ filters.")

plt.figure(figsize=(30,20))
ax_merged = plt.axes(xlabel="Wavelength [Angstrem]", ylabel="Transmission ratio")
ax_merged = plt.plot(wl, ratio)
plt.title("Merged transmission curve.")

print()
\end{lstlisting}

\begin{figure}
\centering
\includegraphics{SDSS_cube_files/SDSS_cube_7_1.png}
\caption{png}
\end{figure}

\begin{figure}
\centering
\includegraphics{SDSS_cube_files/SDSS_cube_7_2.png}
\caption{png}
\end{figure}

\section{Applied transmission curve}\label{applied-transmission-curve}

This is how the spectrum looks like when the photometric transmission
curve is applied. Procedure: 1. Multiply magnitude in each wavelength
with the transmission curve ratio in that wavelength.

\begin{lstlisting}[language=Python]
transmission_ratio, zero_point, softening = photometry.get_photometry_params(spec_flux, spec_wl)

photometric_observed_spectrum_flux = spec_flux * transmission_ratio
photometric_observed_spectrum_flux_sigma =  spec_sigma * transmission_ratio

plt.figure(figsize=(30,20))
ax1 = plt.axes(xlabel="Wavelength [Angstrem]", ylabel="Flux [erg/cm^2/s/Ang]")
ax1.plot(spec_wl, photometric_observed_spectrum_flux)
ax1.fill_between(spec_wl, 
                photometric_observed_spectrum_flux - photometric_observed_spectrum_flux_sigma, 
               photometric_observed_spectrum_flux + photometric_observed_spectrum_flux_sigma, 
              color="orange")
plt.title("Spectrum with applied photometric transmissions")
print()
\end{lstlisting}

\begin{figure}
\centering
\includegraphics{SDSS_cube_files/SDSS_cube_9_1.png}
\caption{png}
\end{figure}

\begin{lstlisting}[language=Python]
spec_zoom_cnt = 2

def plot_cube(res_cube, trans_txt, y_unit):
    for spec in res_cube:
        wl = spec["wl"]
        flux = spec["flux_mean"]
        flux[flux == 0 ] = 'nan'
        flux_sigma = spec["flux_sigma"]
        flux_sigma[flux_sigma == 0 ] = 'nan'
        plt.figure(figsize=(30,20))
        ax = plt.axes(xlabel="Wavelength [Angstrem]", ylabel=y_unit)
        ax.plot(wl, flux)
        ax.fill_between(wl, 
                        flux - flux_sigma, 
                        flux + flux_sigma, 
                        color="orange")
        plt.title("%s, resolution: %s" %(trans_txt, spec["zoom_idx"]))
        
        bins = range(0,len(wl))
        
        plt.figure(figsize=(30,20))
        ax = plt.axes(xlabel="Bin index", ylabel=y_unit)
        ax.plot(bins, flux)
        ax.fill_between(bins, 
                        flux - flux_sigma, 
                        flux + flux_sigma, 
                        color="orange")
        plt.title("%s, bins, resolution: %s" %(trans_txt, spec["zoom_idx"]))

fits_header, spec_cube = photometry.get_multiple_resolution_spectrum(test_spectrum, spec_zoom_cnt,
                                                                     apply_rebin=False, rebin_min=0, rebin_max=0,
                                                                     rebin_samples=0, apply_transmission=False)
fits_header, spec_cube_binned = photometry.get_multiple_resolution_spectrum(test_spectrum, spec_zoom_cnt,
                                                                            apply_rebin=True, rebin_min=3600, rebin_max=10400,
                                                                            rebin_samples=4620, apply_transmission=False)
plot_cube(spec_cube, "Spectrum", "Flux density [erg/cm^2/s/Ang]")
plot_cube(spec_cube_binned, "Spectrum binned", 
          "Flux density [erg/cm^2/s/Ang])")
\end{lstlisting}

\begin{figure}
\centering
\includegraphics{SDSS_cube_files/SDSS_cube_10_0.png}
\caption{png}
\end{figure}

\begin{figure}
\centering
\includegraphics{SDSS_cube_files/SDSS_cube_10_1.png}
\caption{png}
\end{figure}

\begin{figure}
\centering
\includegraphics{SDSS_cube_files/SDSS_cube_10_2.png}
\caption{png}
\end{figure}

\begin{figure}
\centering
\includegraphics{SDSS_cube_files/SDSS_cube_10_3.png}
\caption{png}
\end{figure}

\begin{figure}
\centering
\includegraphics{SDSS_cube_files/SDSS_cube_10_4.png}
\caption{png}
\end{figure}

\begin{figure}
\centering
\includegraphics{SDSS_cube_files/SDSS_cube_10_5.png}
\caption{png}
\end{figure}

\begin{figure}
\centering
\includegraphics{SDSS_cube_files/SDSS_cube_10_6.png}
\caption{png}
\end{figure}

\begin{figure}
\centering
\includegraphics{SDSS_cube_files/SDSS_cube_10_7.png}
\caption{png}
\end{figure}

\begin{figure}
\centering
\includegraphics{SDSS_cube_files/SDSS_cube_10_8.png}
\caption{png}
\end{figure}

\begin{figure}
\centering
\includegraphics{SDSS_cube_files/SDSS_cube_10_9.png}
\caption{png}
\end{figure}

\begin{figure}
\centering
\includegraphics{SDSS_cube_files/SDSS_cube_10_10.png}
\caption{png}
\end{figure}

\begin{figure}
\centering
\includegraphics{SDSS_cube_files/SDSS_cube_10_11.png}
\caption{png}
\end{figure}

\section{Constructing multiple resolution
spectrum}\label{constructing-multiple-resolution-spectrum}

Here we show how the multiple resolution spectrum is constructed, with
the same algorithm as above: 1. In Fluxes without photometric
transmission applied 2. In Magnitudes without photometric transmission
applied 3. In Magnitudes with photomoetric transmission applied

For each of these, we construct the lower resolutions down to the
MIN\_RES setting (Each lower resolution is half of the upper one,
meaning we get spectral resolutions e.g. (4620, 2310, 1155, etc.)

\begin{lstlisting}[language=Python]
spec_zoom_cnt = 4

def plot_cube(res_cube, trans_txt, y_unit):
    for spec in res_cube:
        wl = spec["wl"]
        flux = spec["flux_mean"]
        flux_sigma = spec["flux_sigma"]
        plt.figure(figsize=(30,20))
        ax = plt.axes(xlabel="Wavelength [Angstrem]", ylabel=y_unit)
        ax.plot(wl, flux)
        ax.fill_between(wl, 
                        flux - flux_sigma, 
                        flux + flux_sigma, 
                        color="orange")
        plt.title("%s, resolution: %s" %(trans_txt, spec["zoom_idx"]))

fits_header, spec_cube = photometry.get_multiple_resolution_spectrum(test_spectrum,
                                                                     spec_zoom_cnt,
                                                                     apply_transmission=False)
fits_header, spec_cube_with_transmission = photometry.get_multiple_resolution_spectrum(test_spectrum,
                                                                                       spec_zoom_cnt,
                                                                                       apply_transmission=True)
plot_cube(spec_cube, "Spectrum", "Flux [erg/cm^2/s/Ang]")
plot_cube(spec_cube_with_transmission, "Spectrum, transmission curve applied", 
          "Flux [erg/cm^2/s/Ang])")      
\end{lstlisting}

\begin{figure}
\centering
\includegraphics{SDSS_cube_files/SDSS_cube_12_0.png}
\caption{png}
\end{figure}

\begin{figure}
\centering
\includegraphics{SDSS_cube_files/SDSS_cube_12_1.png}
\caption{png}
\end{figure}

\begin{figure}
\centering
\includegraphics{SDSS_cube_files/SDSS_cube_12_2.png}
\caption{png}
\end{figure}

\begin{figure}
\centering
\includegraphics{SDSS_cube_files/SDSS_cube_12_3.png}
\caption{png}
\end{figure}

\begin{figure}
\centering
\includegraphics{SDSS_cube_files/SDSS_cube_12_4.png}
\caption{png}
\end{figure}

\begin{figure}
\centering
\includegraphics{SDSS_cube_files/SDSS_cube_12_5.png}
\caption{png}
\end{figure}

\begin{figure}
\centering
\includegraphics{SDSS_cube_files/SDSS_cube_12_6.png}
\caption{png}
\end{figure}

\begin{figure}
\centering
\includegraphics{SDSS_cube_files/SDSS_cube_12_7.png}
\caption{png}
\end{figure}

\begin{figure}
\centering
\includegraphics{SDSS_cube_files/SDSS_cube_12_8.png}
\caption{png}
\end{figure}

\begin{figure}
\centering
\includegraphics{SDSS_cube_files/SDSS_cube_12_9.png}
\caption{png}
\end{figure}

\section{In this cell we show how we construct the lower resolutions for
images.}\label{in-this-cell-we-show-how-we-construct-the-lower-resolutions-for-images.}

We use bilinear interpolation to construct the lower resolutions, both
for image measurements and their sigmas, while for each lower
resolution, the sigma is halved

\begin{lstlisting}[language=Python]
img_zoom_cnt = 4


def plot_image_cube(res_cube):
    for img in res_cube:
        plt.figure(figsize=(30,20))
        plt.imshow(img["flux_mean"], cmap='afmhot',  norm=LogNorm())
        plt.colorbar()
        plt.title("Image flux density for res %s" %(str(img["zoom_idx"])))
        #plt.figure(figsize=(30,20))
        #plt.imshow(img["flux_sigma"], cmap='gray')
        #plt.colorbar()
        #plt.title("Image variance for res %s" %(str(img["res"])))
       
        

fits_header, image_cube = photometry.get_multiple_resolution_image(test_image, img_zoom_cnt)

plot_image_cube(image_cube)



        
        
        
\end{lstlisting}

\begin{figure}
\centering
\includegraphics{SDSS_cube_files/SDSS_cube_14_0.png}
\caption{png}
\end{figure}

\begin{figure}
\centering
\includegraphics{SDSS_cube_files/SDSS_cube_14_1.png}
\caption{png}
\end{figure}

\begin{figure}
\centering
\includegraphics{SDSS_cube_files/SDSS_cube_14_2.png}
\caption{png}
\end{figure}

\begin{figure}
\centering
\includegraphics{SDSS_cube_files/SDSS_cube_14_3.png}
\caption{png}
\end{figure}

\begin{figure}
\centering
\includegraphics{SDSS_cube_files/SDSS_cube_14_4.png}
\caption{png}
\end{figure}

\section{Constructing database in HDF5
file}\label{constructing-database-in-hdf5-file}

In this step we load the images and spectra, with their preprocessing
applied above, into HDF5 file. Effectively we combine independently
measured spectra and images into one sparse spectral cube that contains
all the datapoints with comparable values.

Steps involved: 1. Create indexing structure to enable spatial,
temporal, spectral and resolution-wise searching of the datasets. 2.
Preprocess the images and spectra to construct lower resolutions and
uncertainties.

\begin{lstlisting}[language=Python]
import warnings
from astropy.utils.exceptions import AstropyWarning

h5_output_path = "../data/processed/galaxy_small.h5"
input_folder = "../data/raw/galaxy_small"

warnings.simplefilter('ignore', category=AstropyWarning)
command = "python %s/hisscube.py %s %s create" % (module_path, input_folder, h5_output_path)
print("Running command: %s" % command)
!mpiexec -n 8 $command
\end{lstlisting}

\begin{lstlisting}
Running command: python /home/caucau/SDSSCube/hisscube.py ../data/raw/galaxy_small ../data/processed/galaxy_small.h5 create

Image headers: 185it [00:00, 1155.87it/s]

Spectra headers: 11it [00:00, 271.28it/s]

Image metadata: 100%|██████████| 185/185 [00:00<00:00, 250.21it/s]

Spectra metadata: 100%|██████████| 11/11 [00:00<00:00, 479.95it/s]

Image data:   0%|          | 0/185 [00:00<?, ?it/s]^C
\end{lstlisting}

